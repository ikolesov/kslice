%NOTES
%-say more precisely why OMT is good for fire, HS is better for smoke


%\documentclass[12pt,legal]{article} %sets the class type
%\usepackage{spconf,amsmath,epsfig,amsfonts,xspace}



\documentclass[12pt,legal]{article}
\usepackage{theorem,ifthen,algorithm,algorithmic}
\usepackage{amssymb,amsfonts,amsmath,latexsym,dsfont}
\usepackage{fullpage}
\usepackage{graphicx}
\usepackage{mathrsfs}
\usepackage{subfigure}
\usepackage{cite}
\usepackage{fullpage}

\usepackage[citecolor=blue,breaklinks=true,colorlinks]{hyperref}


%\newtheorem{theorem}{Theorem}[section]
%\newtheorem{lemma}[theorem]{Lemma}
%\newtheorem{proposition}[theorem]{Proposition}
%\newtheorem{corollary}[theorem]{Corollary}

%\newenvironment{definition}[1][Definition]{\begin{trivlist}
%\item[\hskip \labelsep {\bfseries #1}]}{\end{trivlist}}
%\newenvironment{example}[1][Example]{\begin{trivlist}
%\item[\hskip \labelsep {\bfseries #1}]}{\end{trivlist}}
%\newenvironment{remark}[1][Remark]{\begin{trivlist}
%\item[\hskip \labelsep {\bfseries #1}]}{\end{trivlist}}

\newcommand{\qed}{\nobreak \ifvmode \relax \else
      \ifdim\lastskip<1.5em \hskip-\lastskip
      \hskip1.5em plus0em minus0.5em \fi \nobreak
      \vrule height0.75em width0.5em depth0.25em\fi}

\newcommand{\ntwo}{\!\!}
\newcommand{\nthree}{\!\!\!}
\newcommand{\nfour}{\!\!\!\!}
\newcommand{\cell}{\mathrm{cell}}
\newcommand{\DIV}{\ensuremath{\mathop{\mathbf{DIV}}}}
\newcommand{\GRAD}{\ensuremath{\mathop{\mathbf{GRAD}}}}
\newcommand{\CURL}{\ensuremath{\mathop{\mathbf{CURL}}}}
\newcommand{\CURLt}{\ensuremath{\mathop{\overline{\mathbf{CURL}}}}}
\newcommand{\nullspace}{\ensuremath{\mathop{\mathrm{null}}}}

\newcommand{\FrameboxA}[2][]{#2}
\newcommand{\Framebox}[1][]{\FrameboxA}
\newcommand{\Fbox}[1]{#1}
\newcommand{\half}{\mbox{\small \(\frac{1}{2}\)}}
\newcommand{\hf}{{\frac 12}}
\newcommand {\HH}  { {\bf H} }
\newcommand{\hH}{\widehat{H}}
\newcommand{\hL}{\widehat{L}}
\newcommand{\bmath}[1]{\mbox{\bf #1}}
\newcommand{\hhat}[1]{\stackrel{\scriptstyle \wedge}{#1}}
%\newcommand{\R}{{\rm I\!R}}
\newcommand {\D} {{\nabla2}}
\newcommand {\sg}{{\hsigma}}
\newcommand{\E}{\vec{E}}
\renewcommand{\H}{\vec{H}}
\newcommand{\J}{\vec{J}}
\newcommand{\dd}{d^{\rm obs}}
\newcommand{\F}{\vec{F}}
\newcommand{\C}{\vec{C}}
\newcommand{\s}{\vec{s}}
\newcommand{\N}{\vec{N}}
\newcommand{\M}{\vec{M}}
\newcommand{\A}{\vec{A}}
\newcommand{\w}{\vec{w}}
\newcommand{\nn}{\vec{n}}
\newcommand{\cA}{{\cal A}}
\newcommand{\cQ}{{\cal Q}}
\newcommand{\cR}{{\cal R}}
\newcommand{\cG}{{\cal G}}
\newcommand{\cW}{{\cal W}}
\newcommand{\hsig}{\hat \sigma}
\newcommand{\hJ}{\hat \J}
\newcommand{\hbeta}{\widehat \beta}
\newcommand{\lam}{\lambda}
\newcommand{\dt}{\delta t}
\newcommand{\kp}{\kappa}
\newcommand {\lag} { {\cal L}}
\newcommand{\zero}{\vec{0}}
\newcommand{\Hr}{H_{red}}
\newcommand{\Mr}{M_{red}}
\newcommand{\mr}{m_{ref}}
\newcommand{\thet}{\ensuremath{\mbox{\boldmath $\theta$}}}
\newcommand{\curl}{\ensuremath{\nabla\times\,}}
\renewcommand{\div}{\nabla\cdot\,}
\newcommand{\grad}{\ensuremath{\nabla}}
\newcommand{\dm}{\delta m}
\newcommand{\gradh}{\ensuremath{\nabla}_h}
\newcommand{\divh}{\nabla_h\cdot\,}
\newcommand{\curlh}{\ensuremath{\nabla_h\times\,}}
\newcommand{\curlht}{\ensuremath{\nabla_h^T\times\,}}
\newcommand{\Q}{\vec{Q}}
\newcommand{\V}{\vec{V}}

\newcommand{\bfA}{{\bf A}}
\newcommand{\bfB}{{\bf B}}
\newcommand{\bfC}{{\bf C}}
\newcommand{\bfD}{{\bf D}}
\newcommand{\bfE}{{\bf E}}
\newcommand{\bfF}{{\bf F}}
\newcommand{\bfG}{{\bf G}}
\newcommand{\bfH}{{\bf H}}
\newcommand{\bfI}{{\bf I}}
\newcommand{\bfJ}{{\bf J}}
\newcommand{\bfK}{{\bf K}}
\newcommand{\bfL}{{\bf L}}
\newcommand{\bfM}{{\bf M}}
\newcommand{\bfN}{{\bf N}}
\newcommand{\bfO}{{\bf O}}
\newcommand{\bfP}{{\bf P}}
\newcommand{\bfQ}{{\bf Q}}
\newcommand{\bfR}{{\bf R}}
\newcommand{\bfS}{{\bf S}}
\newcommand{\bfT}{{\bf T}}
\newcommand{\bfU}{{\bf U}}
\newcommand{\bfV}{{\bf V}}
\newcommand{\bfW}{{\bf W}}
\newcommand{\bfX}{{\bf X}}
\newcommand{\bfY}{{\bf Y}}
\newcommand{\bfZ}{{\bf Z}}

\newcommand{\bfa}{{\bf a}}
\newcommand{\bfe}{{\bf e}}
\newcommand{\bfh}{{\bf h}}
\newcommand{\bfj}{{\bf j}}
\newcommand{\bfs}{{\bf s}}
\newcommand{\bfx}{{\bf x}}
\newcommand{\bfu}{{\bf u}}
\newcommand{\bfq}{{\bf q}}
\newcommand{\bfn}{{\bf n}}

\newcommand{\vu}{{\vec {\bf u}}}


\newcommand{\CL}{{\cal L}}
\newcommand{\CO}{{\cal O}}
\newcommand{\CP}{{\cal P}}
\newcommand{\CQ}{{\cal Q}}
\newcommand{\CH}{{\mathscr H}}

\newcommand{\Fc}{{\cal F}}
\newcommand{\bfphi}{{\boldsymbol \phi}}
\newcommand{\bfmu}{{\boldsymbol \mu}}

\newcommand{\Ex}{{\rm E}}
\newcommand{\Var}{{\rm Var}}

\newcommand{\diag}{\mathrm{diag}\,}

\newcommand{\wht}{\widehat}
\newcommand{\rf}{\rm ref}

\newcommand{\kk}{\vec{k}}
\newcommand{\bo}{d^{\rm obs}}
\newcommand{\xx}{\vec{x}}
\newcommand{\what}{\widehat}
\renewcommand{\theequation}{\arabic{section}.\arabic{equation}}

\newcommand{\trace}{\mathrm{trace}}
\newcommand{\st}{\mathrm{s.t.}}

\newcommand{\vb}{\mathbf}
\newcommand{\R}{\ensuremath{\mathds{R}}}
\newcommand{\abs}[1]{\ensuremath{\left|#1\right|}}
\newcommand{\norm}[1]{\ensuremath{\left\|#1\right\|}}
\newcommand{\iprod}[1]{\ensuremath{\left\langle#1\right\rangle}}

\newcommand{\dwu}{\delta \widehat u}
\newcommand{\dwx}{\delta \widehat x}
\newcommand{\Mu}{M_{\mu}}

\newcommand{\la}{\left\langle}
\newcommand{\ra}{\right\rangle}

\newtheorem{example}{Example}
\newtheorem{lemma}{Lemma}
\newtheorem{theorem}{Theorem}
\newtheorem{proof}{Proof}

\newcommand{\be}{\begin{equation}
}

\newcommand{\ee}{\end{equation}}
\newcommand{\ws}{(\kappa^2 + \tau^2)^{1/2}}

\renewcommand{\div}{\nabla\cdot\,} %divergence symbol
%\newcommand{\hf}{{\frac 12}} %1/2
%\newcommand{\grad}{\ensuremath{\nabla}} %gradient symbol

\DeclareMathOperator*{\argmin}{argmin} % so doesn't show up italicized

   %%%%%%%%%%%%%%%%%%%%%%%%%%%% Setting to control figure placement
   % These determine the rules used to place floating objects like figures
   % They are only guides, but read the manual to see the effect of each.
   \renewcommand{\topfraction}{.9}
   \renewcommand{\bottomfraction}{.9}
   \renewcommand{\textfraction}{.1}



%% separate entries in the bibliography
%\usepackage[numbers]{natbib} \setlength{\bibsep}{0ex} % no separation
%
%\usepackage{color} \definecolor{darkblue}{rgb}{.1,.1,.5}
%\usepackage[colorlinks,citecolor=darkblue,linkcolor=darkblue,urlcolor=darkblue]{hyperref}
%\usepackage[numbers]{natbib} \setlength{\bibsep}{0ex}
%\usepackage[all]{hypcap} % hyperref/caption fix

%\usepackage[dvips]{graphicx} %graphics

%\title{Detecting Smoke and Flame in Video Using a Neural Network Classifier}
%\name{Ivan Kolesov \footnote{\thanks{Kolesov, Karasev and Tannenbaum are with
%        the School of Electrical and Computer Engineering at the
%        Georgia Institute of Technology, Atlanta, GA 30332. Emails: ivan.kolesov@gatech.edu,
%        pkarasev@gatech.edu, tannenba@ece.gatech.edu. Tannenbaum is also
%        with the Department of EE, Technion, Israel where he is
%        supported by a Marie Curie Grant through the EU. ****Write something about Eldad*****}}
%        \quad Peter Karasev \quad Eldad Haber \quad Allen Tannenbaum }
%\address{}
%%\address{%\texttt{pkarasev@gatech.edu}\\
%%School of Electrical and Computer Engineering\\
%%Georgia Institute of Technology, Atlanta, Georgia}




\begin{document}
\title{Parametric Surface fit to Mesh}
\author{Peter Karasev, Matias Perez}
\maketitle

\section{Triangulated Mesh Data}

Suppose we have a triangulated mesh in three-dimensional space:
\begin{eqnarray} \label{mesheq}
f_i = (j_i, \ k_i, \ \ell_i) \\ \label{mesheq2}
v_p = (x_p,\ y_p,\ z_p) 
\end{eqnarray}
\noindent with $i=1 \cdots N$, $N$ the number of triangle faces in the mesh, and $p = 1 \cdots M$, $M$ the number of vertices in the data\footnote{all $M$ vertices are not necessarily used by the faces, and there could be duplicates}. The $i$-th face consists of vertices $(v_j,v_k,v_{\ell})_i$. \\

\section{The Question: Fit a parameterized differentiable surface to the data?}

Recall that differential geometry for surfaces in 3D is phrased as a surface taking a differentiable function $\mathbf{x}$ mapping an open set to an open set:  $\mathbf{x}: U \subset \mathbb{R}^2 \rightarrow \Omega \subset \mathbb{R}^3$.

Then one looks at $\mathbf{x}(u^1,u^2) = \big( x(u^1,u^2), y(u^1,u^2), z(u^1,u^2) \big)$ and derivatives $\partial{x}/\partial u^i$, and so on. 

Notice that getting to this \textit{nice} description from \autoref{mesheq}-\autoref{mesheq2} is not straightforward. How do you make such a parameterization, that is differentiable, given only a bunch of vertices and faces defining connectivity? 

Task: determine how to form a differentiable parametric surface description $\mathbf{x}(u^1,u^2)$ given data of the form in \autoref{mesheq}-\autoref{mesheq2}. It might be defined to exactly fit the vertices, or to be 'close' if smoothing is desired. 

\section{Purpose}

To step back for a moment, let us think about why having a parameterized surface would be nice. If such a thing existed, all of the surface derivatives and geometrical quantities could be written down in closed form, rather than as a numerical fit. 

Furthermore, a problem that keeps coming up in the current mesh contour segmentation work is local minima of geometrical quantities. For example, there can arise a patch of high curvature due to a jagged piece of mesh, thus throwing off a contour evolution. It would be great if somehow a 'regularization' factor could be included in the construction of a parametric surface, thus keeping only large-scale geometric features. 

Lastly is data compression- if we only need a relatively small number of basis functions with which to do large-scale geometric computations, great. 

\section{Observations and Thoughts}

This should be rather easy if the surface is assumed to be either the graph of a function $\mathbf{x}(u^1,u^2) = (u^1,u^2,f(u^1,u^2))$. \\

It is just slightly harder if it is a hypersurface of the form $\mathbf{x}(u^1,u^2) = g^{-1}(0)$ where there is some function $g: \mathbb{R}^3 \rightarrow \mathbb{R}$ and the surface is g's zero level set. Example: $g(x,y,z) = x^2+y^2+z^2-1$ in which case $\mathbf{x}$ is the unit sphere. \\

Unfortuneately, given observations of how segmentation methods in medical imaging create data, a practical / useable approach has to allow for surfaces that fit neither of these cases. For example, something topologically similar to a torus cut in half. \\

Some methods to consider: 
\begin{itemize}
	\item {least-squares fitting of basis functions to the given vertices. A basis that makes sense has to be chosen, if it is to work for general meshes}
	\item { wavelet-shrinkage : this relates to the first item, it seems to be similar and there's some experience with this in the lab} \\
	\item { calculus of variations: we are trying to determine an unknown differentiable function with some constraints (fitting the data), a variational method might make sense }
\end{itemize}

\textbf{***Warning*** } I have not yet done a sufficient amount of background reading on this question; it may be already solved. In fact it is definately solved for the \textit{graph of a function} case (interpolation!), and probably somewhere for the \textit{hypersurface} case. The more general case would be harder to find, but there has to be a period of 4-8 hours of digging through papers before one concludes that it is or is not solved already.

\end{document}